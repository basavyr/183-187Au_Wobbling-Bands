\documentclass[12pt, a4paper]{article}
\usepackage[top=2cm, bottom=2.5cm, left=2.5cm, right=2.5cm]{geometry}
\usepackage[utf8]{inputenc}
\usepackage{graphicx}
\usepackage{amsmath}
\usepackage{mathptmx}
\usepackage{physics}
\usepackage{xcolor}
\bibliographystyle{unsrt}

\title{New results concerning the wobbling properties of $^{183,187}Au$}
\author{Robert Poenaru}
\date{\today}


\begin{document}
\maketitle

\section{Introduction}

Two wobbling sequences have been identified in $^{183}$Au by Nandi et. al. \cite{nandi2020}. One sequence has two bands with states of negative parity (built on top of the odd $h_{9/2}$ proton) and two bands with states of positive parity (built on top of the odd $i_{13/2}$ proton). Both sequences are considered to have $n_w=0$ for the \textit{yrast} band and $n_w=1$ for the one-phonon wobbling band.


\begin{figure}[ht]
    \centering
    \includegraphics[scale=0.3]{figs/negative_Au183.png}\hspace{2cm}
    \includegraphics[scale=0.3]{figs/positive_Au183.png}\hspace{1.5cm}
    \includegraphics[scale=0.3]{figs/spectrum_Au187.png}
    \caption{\textbf{Left:} $^{183}$Au: negative parity states based on $j=9/2$.\textbf{Middle:} $^{183}$Au: positive parity states based on $j=13/2$. \textbf{Right:} The wobbling structure in $^{187}$Au.}
    \label{au_wobbling_bands}
\end{figure}

On the other hand, Sensharma et. al. \cite{sensharma2020} has confirmed wobbling motion in $^{187}$Au, with the identification of two such bands, show in figure \ref{au_wobbling_bands}.

\section{Numerical application}

By using the same formalism as the one applied for $^{163}$Lu in \cite{poenaru2021parity}. Namely, the both positive and negative wobbling sequences from $^{183}$Au were described with the same analytical expressions for the \textit{excitation energies}.

\begin{align}
    E_\text{exc}(I)=\epsilon_j+\mathcal{H}_\text{min}(I)+\Omega_1^I\left(n_{w_1}+1\right)+\Omega_2^I\left(n_{w_2}+1\right)\ ,
    \label{wobbling_energies}
\end{align}
such that $E_\text{exc}(I)=\mathcal{F}(I,j;\mathcal{P})$, where $\mathcal{P}=\left[\mathcal{I}_1,\mathcal{I}_2,\mathcal{I}_3,V,\gamma\right]$ is the \textbf{free parameter set}.

The wobbling frequencies $\Omega_1$ and $\Omega_2$ are the solutions of the algebraic equation:

\begin{align}
    \Omega^4+B\Omega^2+C=0
\end{align}
and 
\begin{align}
    \Omega_1=\sqrt{\frac{1}{2}\left(-B{\color{red}\textbf{+}}\sqrt{B^2-4C}\right)}\\
    \Omega_2=\sqrt{\frac{1}{2}\left(-B{\color{red}\textbf{-}}\sqrt{B^2-4C}\right)}.
\end{align}

\section{Coupling schemes}

\subsection{$^{183}$Au - positive parity}

The spin states with positive parity are created by the coupling of the even-even rotor $\vec{R}$ with the odd-proton $i_{j=13/2}$. As such, the yrast band emerges from a rotor with even angular momentum $\vec{R}=0,2,4,\dots$, while the first excited wobbling band emerges from the coupling of the same $j$ but with a rotor with odd spin sequence $\vec{R}=1,3,5\dots$.

Within Eq. \ref{wobbling_energies}, the two wobbling phonon numbers (i.e., $n_{w_1}$ and $n_{w_2}$) are $(0,0)$ and $(1,0)$ for the yrast and excited bands, respectively.



\newpage

\bibliography{references}

\end{document}